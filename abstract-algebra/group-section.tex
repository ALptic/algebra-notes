
\section{群的概念与基本性质}
\label{sec:group}

\subsection{概念}
\label{sec:concept-of-group}



\begin{definition}
  设$G$是一个非空集合,其上定义了一个二元运算$\odot$,即对于任意$a,b \in G$,存在$c \in G$,使得$a\odot b=c$,如果运算$\odot$满足结合律,则称集合$G$及其上的二元运算$\odot$为一个\emph{半群},记作$<G, \odot>$。
\end{definition}

为了避免书写繁琐,由于实数上的乘法运算可以省去运算符不写,即把$a\times b$写成$ab$,这里也把半群中的二元运算称为乘法,并把$a\odot b$简写为$ab$,但是这里所称的乘法只是一个名称而已,与实数的乘法没有任何关系,只是为了省写运算符罢了,以后就直接称半群上的乘法了。

\begin{definition}
  设$G$是一个非空集合,其上有定义了称为乘法的二元运算,如果存在一个元素$e \in G$,使得对于任意$a \in G$,都成立$ea=a$,则称$e_l$是集合$G$关于这个运算的\emph{左幺元},如果把等式改成$ae=a$,则称$e$是\emph{右幺元},如果左幺元与右幺元是同一个元素,则称它是\emph{幺元}。
\end{definition}

\begin{definition}
  仍设$G$是一非空集合,其上定义了称为乘法的二元运算,而且它有左幺元$e$,设$a \in G$,如果存在$b \in G$,使得$ba=e$,则称$b$是$a$的\emph{左逆元},在这定义中把左幺元换成右幺元,并把等式$ba=e$换成$ab=e$,则得到\emph{右逆元}的定义。如果左幺元与右幺元是同一元素,即幺元,并且左逆元与右逆元是同一个,则称$b$是$a$的\emph{逆元}.
\end{definition}

在有了幺元和逆元的概念后,就可以给出群的概念了。

\begin{definition}
  设$G$关于其上的二元乘法是一个半群,如果还满足以下两个条件,则称集合$G$关于此乘法运算构成一个\emph{群}:
  \begin{enumerate}
  \item 存在$e \in G$,使得对于任意$a \in G$,都有$ea=ae=a$,此时称$e$是幺元。
  \item 任意$a \in G$,都存在$b \in G$,使得$ab=ba=e$,此时称$b$是$a$的逆元。
  \end{enumerate}
\end{definition}

\begin{definition}
  设非空集合$G$关于其上的乘法是一个群,如果此乘法还满足交换律,则称这个群是\emph{交换群},或者\emph{阿贝尔(Abel)群}.
\end{definition}

\begin{definition}
  群中的元素个数称为群的\emph{阶},元素个数有限的群称为\emph{有限群},元素个数无限的群称为\emph{无限群}.
\end{definition}

实际上群的定义还可以减弱,只要存在左幺元和左逆元就可以了,或者存在右幺元和右逆元,这都是等价的,现在来讨论这个问题,先证明一个关于消去律的结论.

\begin{theorem}
  \label{theorem:left-identity-reverse-ele-so-left-cancellation}
  设非空集合$G$关于其上的乘法是一个半群,如果存在左幺元并且$G$中任意元素都存在关于左幺元的左逆元,则这乘法满足左消去律,即由$ab=ac$能得出$b=c$.同样的,由右幺元及右逆元也能得出右消去律。
\end{theorem}

\begin{proof}[证明]
  假定三个元素$a$、$b$、$c$符号等式$ab=ac$,用$a$的左逆元$d$左乘等式两端得$d(ab)=d(ac)$,由结合律得$(da)b=(da)c$,即$eb=ec$,于是$b=c$.右消去律的情况同理。
\end{proof}

\begin{theorem}
  设非空集合$G$关于其上的乘法是一个半群,如果存在左幺元和左逆元,则它也必然存在右幺元和右逆元,并且左右幺元相等,左右逆元相等,反之亦然。
\end{theorem}

\begin{proof}[证明]
  设左幺元是$e$,先证明它也是右幺元,即对任意$a \in G$,需要证明$ae=a$,设$a$的左逆元为$b$,即$ba=e$,有$b(ae)=(ba)e=ee=e=ba$,根据 \autoref{theorem:left-identity-reverse-ele-so-left-cancellation},按左消去律即得$ae=a$,即$e$也是右幺元。

  再证明$a$的左逆元$b$同时也是它的右逆元,因为$be=b=eb=(ba)b=b(ab)$,按左消去律即得$ab=e$,即$ab=ba=e$.即得证。
\end{proof}

有了这个定理,群的定义就可以简化了,只要存在左幺元以及任何元素都存在关于左幺元的左逆元就够了,这称为群的第二定义。

\subsection{基本性质}
\label{sec:basic-properties-of-group}

\begin{theorem}[群的基本性质定理]
  设非空集合$G$关于其上的乘法运算构成一个群,则
  \begin{enumerate}
  \item 左右幺元相等,幺元是唯一的,记作$e$.
  \item 任意元素的左右逆元素相等且唯一,记作$a^{-1}$。
  \item 乘法运算满足消去律,包括左消去律和右消去律.
  \item 对于$G$中任取两个元素$a$和$b$(可以相等),方程$ax=b$及$xa=b$都有唯一解.
  \end{enumerate}
\end{theorem}

这里的性质与前文所给出群的定义有部分重复,所以这里是采用的群的第二定义。

\begin{proof}[证明]
  1. 前文已经证明了左幺元同时也是右幺元,只需再证幺元的唯一性,设$e_1$和$e_2$都是幺元,则$e_1=e_1e_2=e_2$,即幺元是唯一的。

  2. 同样,任意元素的左逆元同时也是它的右逆元,前文已经证明了这点,只要再证明逆元的唯一性,设$b_1$和$b_2$都是$a$的逆元,则$b_1=b_1e=b_1(ab_2)=(b_1a)b_2=eb_2=b_2$,所以逆元也是唯一的。

  3. 消去律前文已经证明了,由左幺元和左逆元可得出左消去律,由右幺元和右逆元可得出右消去律。

  4. 对于方程$ax=b$,两边同时左乘$a$的逆元$a^{-1}$,得$x=a^{-1}b$,即为其唯一解,对于方程$xa=b$,两边同时右乘$a$的逆元$a^{-1}$,即得唯一解$x=ba^{-1}$,需要注意的是这两个方程的解并不一定相等,因为乘法并不一定满足交换律。
\end{proof}


%%% Local Variables:
%%% mode: latex
%%% TeX-master: "../algebra-note"
%%% End:
